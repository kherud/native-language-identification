\documentclass[bachelor,german]{info1thesis}

\usepackage[utf8]{inputenc}
%\usepackage[T1]{fontenc}
%\usepackage{fontspec}
%\usepackage{lmodern} TODO: braucht man das?

\usepackage{tabularx}
\usepackage{ltablex}

\usepackage{path}
\usepackage{color}
\usepackage{booktabs}
\usepackage{multirow}

%\usepackage[disable,colorinlistoftodos]{luatodonotes}


%%%%%%%%%%%%%%%%%%%%%%%%%%%%%%%%%%%%%%%%%%%%%%%%%%%%%%%%%%%%%%%%%%%%%%%%%%%%%%%%
%%% Titelseite -- hier Titel und Autorennamen eintragen
\title{Machine Learning for Natural Language Processing} % Geben Sie hier den Titel Ihrer Arbeit an.
\subtitle{am Lehrstuhl für Informatik X}
\author{Konstantin Herud\and Thomas Schaffroth\and Maximilian Meißner} % Geben Sie Ihren Namen an.
%\date{Eingereicht am \abgabedatum}
\titlehead{Julius-Maximilians-Universität Würzburg\\
Institut für Informatik\\
Lehrstuhl für Informatik X\\
Data Science}
\supervisors{Prof. Dr. Andreas Hotho\and Daniel Schlör\and Albin Zehe\and Konstantin Kobs\and Tobias Koopmann}

\begin{document}
%%%%%%%%%%%%%%%%%%%%%%%%%%%%%%%%%%%%%%%%%%%%%%%%%%%%%%%%%%%%%%%%%%%%%%%%%%
%%%%%%%%%%%%% Bitte nur ab hier Änderungen vornehmen %%%%%%%%%%%%%%%%%%%%%

\begin{abstract}
    Dieses Dokument soll Studenten an unserem Lehrstuhl bei der Erstellung
    ihrer Abschlussarbeit unterstützen.
    Wir zeigen eine beispielhafte Gliederung einer Arbeit und beschreiben
    die Inhalten der einzelnen Kapitel.
    Zusätzlich geben wir an vielen Stellen auch Hinweise zur Benutzung von
    \LaTeX\ für die Erstellung der Arbeit.
    Im Anhang~\ref{appendix:orga} geben wir ein paar Hinweise zum Ablauf der
    Betreuung von Abschlussarbeiten an unserem Lehrstuhl.

    \paragraph{Zur Handhabung dieses Pakets.}
    In diesem Paket sind Vorlagen für verschiedene Dokumenttypen enthalten, die
    sie als Ausgangspunkt für ihre Arbeit verwenden können.
    Es gibt jeweils Vorlagen für deutsche und englische Arbeiten.
    \begin{itemize}
        \item \verb+template_thesis_de.tex+, \verb+template_thesis_en.tex+:
            Vorlage für Bachelorarbeit bzw. Masterarbeit
        \item \verb+template_seminar_de.tex+, \verb+template_seminar_en.tex+:
            Vorlage für Seminarausarbeitungen und Praktikumsberichte
    \end{itemize}
    Der Quelltext zu diesem Leitfaden ist ebenfalls im Paket enthalten.
    Diesen können Sie als praktisches Beispiel dafür verwenden, wie diese
    Dokumentenklasse angewandt wird.

    \paragraph{Inhalt der Zusammenfassung.}
    Schreiben Sie hier eine Zusammenfassung der Arbeit, vergleichbar mit dem Abstract auf wissenschaftlichen Papers.
    Sie dient dem Leser dazu, einen groben Überblick über die Inhalte zu gewinnen (Problemstellung, verwendeter Lösungsansatz, ggf.\ experimentelle Ergebnisse, gewonnene Erkentnisse).
    Der Umfang soll ca.\ eine halbe Seite betragen.
    Für Seminararbeiten ist diese Zusammenfassung nicht erforderlich.
    
    \emph{Achtung:} Bei Arbeiten auf Englisch fordern die
    Prüfungsordnungen, dass es eine deutsche Zusammenfassung gibt.
    Schreiben Sie in diesem Fall eine englische \emph{und} eine deutsche Zusammenfassung (mit dem gleichen Inhalt).
    Die passenden \LaTeX-Befehle dafür finden Sie in den englischsprachigen
    Vorlagen.

    \paragraph{WARNUNG:} 
    Die vorliegende Version des Leitfadens ist eine \textcolor{red}{Vorabversion}, die noch nicht vollständig ist.
    Sie bezieht sich größtenteils auf die Ausarbeitung von Bachelor- und Masterarbeiten; Seminararbeiten unterscheiden sich davon etwas in Aufbau und Inhalt.

%    \vspace{3em}
%    \textbf{TODOs für die Titelseite:}
%    \todo{Soll der Name des Lehrstuhls für englische Arbeiten übersetzt werden?}
\end{abstract}

\thesistableofcontents






%%%%%%%%%%%%%%%%%%%%%%%%%%%%%%%%%%%%%%%%%%%%%%%%%%%%%%%%%%%%%%%%%%%%%%%%%%%%%%%%

\chapter{Methodik}
\section{Task 2: Klassifikation}

\begin{itemize}
\item $z$: Zeilen im Dokument
\item $t$: Tokens pro Zeile (Padding kürzerer Zeilen)
\item $d$: Dimension des Modells (z.\,B. 150)
\item $c$: Anzahl Klassen
\end{itemize}

\begin{align}
Y_{ztd} &= \text{LSTM}_t(X_{ztd}) \\
A_{zt} &= \text{softmax}_t\left(\sum_d Y_{ztd} W_{d}^1\right) \\
Y_{zd} &= \sum_t Y_{ztd} A_{zt} \\
A_{z} &= \text{softmax}_z\left(\sum_d Y_{zd} W_{d}^2\right) \\
Y_{d} &= \sum_z Y_{zd} A_z \\
Y_c &= Y_d W_{dc}^3
\end{align}


\begin{equation}
W_{L,C_i} = \left(\frac{|C_{\max}|}{|C_i|}\right)^\alpha,\;\;\;\alpha = 0.6
\end{equation}

% Please add the following required packages to your document preamble:
% \usepackage{booktabs}

\begin{figure}[h!]
\centering
\includegraphics[width=\textwidth]{img/attention-visualized.png}
\end{figure}

\begin{table}[]
\centering
%\scalebox{1.3}{
\begin{tabular}{@{}clrrrrrrrrrrrr@{}}
\multicolumn{1}{l}{}       & \multicolumn{13}{c}{Voraussage}                                   \\ \cmidrule(l){2-14} 
\multicolumn{1}{l}{}       &     & AU & CA & CN  & FR & DE & IN & IL & JP & SG & CH & UK & USA \\ \cmidrule(l){2-14} 
\multirow{11}{*}{\rotatebox[origin=c]{90}{Wahrheit}} & AU  & 6  &    & 8   &    & 1  &    &    &    &    &    & 7  & 6   \\
                           & CA  & 1  & 8  & 4   &    &    &    &    & 2  &    & 1  & 7  & 32  \\
                           & CN  & 1  & 1  & 188 &    & 1  &    &    & 2  & 1  &    & 1  & 17  \\
                           & FR  &    &    & 1   & 36 & 3  &    &    & 1  &    &    & 2  & 4   \\
                           & DE  &   & 2  & 2   & 2  & 47 &    &    & 1  &    &    & 12 & 11  \\
                           & IN  &    &    & 1   &    &    & 6  &    &    &    &    & 2  & 11  \\
                           & IL  &    & 1  &     & 1  &    &    & 14 & 2  &    &    &    & 10  \\
                           & JP  &    &    & 2   & 1  &    &    &    & 45 &    & 1  & 1  & 6   \\
                           & SG  &    & 1  & 6   &    &    &    &    &    & 9  &    & 2  & 8   \\
                           & CH  &    &    & 3   &    & 7  &    &    &    &    & 4  & 4  & 8   \\
                           & UK  & 3  & 5  & 1   & 6  & 1  & 2  &    & 1  &    & 2  & 66 & 18  \\
                           & USA & 3  & 13 & 38  & 17 & 6  & 1  & 2  & 8  & 2  & 3  & 32 & 746 \\ \cmidrule(l){2-14} 
\end{tabular}
%}
\caption{Konfusionsmatrix der Klassen Australien (AU), Canada (CA), China (CN), Frankreich(FR), Deutschland (DE), Indien (IN), Israel (IL), Japan (JP), Singapur (SG), Schweitz (CH), Großbritannien (UK), Amerika (USA). Leere Zellen implizieren Nullen.}
\end{table}



%%%%%%%%%%%%%%%%%%%%%%%%%%%%%%%%%%%%%%%%%%%%%%%%%%%%%%%%%%%%%%%%%%%%%%%%%%%%%%%%
\thesisbibliography
\bibliography{report}



\appendix


\chapter{Formelles/LaTeX}


%\clearpage
%\pdfbookmark[0]{ToDo-Liste}{todos}
%\listoftodos

\end{document}
