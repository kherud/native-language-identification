\documentclass[runningheads,a4paper,12pt]{uwsese}

%ACHTUNG: Diesen Import für englische Arbeiten entfernen.
\usepackage[ngerman]{babel}

\usepackage[utf8]{inputenc}
\usepackage{amssymb}
\usepackage{amsmath}
\setcounter{tocdepth}{3}
\usepackage{graphicx}
\usepackage{array}
\usepackage{ragged2e}
\usepackage{arydshln}
\usepackage{xcolor}
\usepackage{booktabs}
\usepackage{multirow}
\newcolumntype{P}[1]{>{\RaggedRight\hspace{0pt}}p{#1}}
%\newcolumntype{P}[1]{>{\raggedright\arraybackslash}p{#1}}


\usepackage{url}
\urldef{\mailsa}\path|dozent.ls2.informatik@uni-wuerzburg.de| 
\newcommand{\keywords}[1]{\par\addvspace\baselineskip
\noindent\keywordname\enspace\ignorespaces#1}

\begin{document}

\mainmatter 

% first the title is needed
\title{Bearbeitungshinweise\\ Seminar Software Engineering}

% a short form should be given in case it is too long for the running head
\titlerunning{Bearbeitungshinweise Seminar Software Engineering}


\author{Informatik LS2 Dozent}
%
\authorrunning{LS2 Dozent}
% (feature abused for this document to repeat the title also on left hand pages)

% the affiliations are given next
\institute{University of W\"urzburg,
Germany\\
\mailsa\\
}


\toctitle{Interpolation based Modelling of Power and Performance}
\tocauthor{J. von Kistowski and S. Kounevs}
\maketitle


\begin{abstract}
	Hier kommt der Abstract hin. Der Abstract behandelt alle Unterpunkte, welche auch in der Introduction behandelt werden (s. Section~\ref{sub:intro}) und liefert damit eine Zusammenfassung über die Seminararbeit als Ganzes. Im Gegensatz zur Zusammenfassung ist der Abstract sehr kurz gehalten und hat ca. ein bis zwei Sätze pro Unterpunkt.
\end{abstract}


\section{Introduction}

\section{Methodik}
\subsection{Task 2: Klassifikation}

\begin{itemize}
\item $s$: Zeilen im Dokument,
\item $t$: Tokens pro Zeile (Padding gemäß längster Zeile)
\item $d$: Dimension des Modells (z.\,B. 150)
\item $c$: Anzahl Klassen
\end{itemize}

\begin{align}
X_{ztd} &= \text{LSTM}_t(X) \\
A_{zt} &= \text{softmax}_t\left(\sum_d X_{ztd} W_{d}^1\right) \\
X_{zd} &= \sum_t X_{ztd} A_{zt} \\
A_{z} &= \text{softmax}_z\left(\sum_d X_{zd} W_{d}^2\right) \\
X_{d} &= \sum_z X_{zd} A_z \\
Y_c &= X_d W_{dc}
\end{align}


\begin{equation}
W_{L,C_i} = \left(\frac{|C_{\max}|}{|C_i|}\right)^\alpha,\;\;\;\alpha = 0.6
\end{equation}

% Please add the following required packages to your document preamble:
% \usepackage{booktabs}

\begin{figure}[h!]
\centering
\includegraphics[width=\textwidth]{img/attention-visualized.png}
\end{figure}

\begin{table}[]
\centering
\begin{tabular}{@{}clllllllllllll@{}}
\multicolumn{1}{l}{}       & \multicolumn{13}{c}{Voraussage}                                   \\ \cmidrule(l){2-14} 
\multicolumn{1}{l}{}       &     & AU & CA & CN  & FR & DE & IN & IL & JP & SG & CH & UK & USA \\ \cmidrule(l){2-14} 
\multirow{11}{*}{\rotatebox[origin=c]{90}{Wahrheit}} & AU  & 6  &    & 8   &    & 1  &    &    &    &    &    & 7  & 6   \\
                           & CA  & 1  & 8  & 4   &    &    &    &    & 2  &    & 1  & 7  & 32  \\
                           & CN  & 1  & 1  & 188 &    & 1  &    &    & 2  & 1  &    & 1  & 17  \\
                           & FR  &    &    & 1   & 36 & 3  &    &    & 1  &    &    & 2  & 4   \\
                           & DE  &   & 2  & 2   & 2  & 47 &    &    & 1  &    &    & 12 & 11  \\
                           & IN  &    &    & 1   &    &    & 6  &    &    &    &    & 2  & 11  \\
                           & IL  &    & 1  &     & 1  &    &    & 14 & 2  &    &    &    & 10  \\
                           & JP  &    &    & 2   & 1  &    &    &    & 45 &    & 1  & 1  & 6   \\
                           & SG  &    & 1  & 6   &    &    &    &    &    & 9  &    & 2  & 8   \\
                           & CH  &    &    & 3   &    & 7  &    &    &    &    & 4  & 4  & 8   \\
                           & UK  & 3  & 5  & 1   & 6  & 1  & 2  &    & 1  &    & 2  & 66 & 18  \\
                           & USA & 3  & 13 & 38  & 17 & 6  & 1  & 2  & 8  & 2  & 3  & 32 & 746 \\ \cmidrule(l){2-14} 
\end{tabular}
\caption{Konfusionsmatrix der Klassen Australien (AU), Canada (CA), China (CN), Frankreich(FR), Deutschland (DE), Indien (IN), Israel (IL), Japan (JP), Singapur (SG), Schweitz (CH), Großbritannien (UK), Amerika (USA)}
\end{table}



%\newpage
\bibliographystyle{splncs03}
\bibliography{sesebib}

\end{document}
